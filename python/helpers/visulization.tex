%& -jobname=vis.pdf
%%%%%%%%%%%%%%%%%%%%%%%%%%%%%%%%%%%%%%%%%%%%%%%%%%%
% Place this file inside python/benchmark_out/notes
%%%%%%%%%%%%%%%%%%%%%%%%%%%%%%%%%%%%%%%%%%%%%%%%%%%
\documentclass{article}
\usepackage[margin=1in]{geometry}
\usepackage{graphicx}
\usepackage{fancyhdr}
\usepackage[yyyymmdd,hhmmss]{datetime}
\usepackage{pgffor}
\pagestyle{fancy}
\lhead{\today, \currenttime}

\renewcommand{\headrulewidth}{0pt}
\setlength{\parindent}{0pt} 
\renewcommand{\arraystretch}{1.5}

%%%%%%%%%%%%%%%%%%%%%%%%%%%%%%%%%%%%%%%%%%%%%%%%
% Change those variables to display helpful info
%%%%%%%%%%%%%%%%%%%%%%%%%%%%%%%%%%%%%%%%%%%%%%%%
\def\layers{32} % Set resolution (number of voxel layers)
\def\cgs{20} % Include all CGs less than this number for each simulation
\def\method{N=3} % Method name
\def\enrich{none} % Enrich method name 
\newcounter{figs}
\setcounter{figs}{0}

\begin{document}
    \begin{center}
        \Large Visualization Layer by Layer
    \end{center}

    \begin{table}[ht]
        \centering
        \begin{tabular}{p{5cm} c}
            Method & \method \\
            Voxel height & \layers \\
            Enrich method& \enrich
        \end{tabular}    
    \end{table}
    
    \bigskip
        \foreach \x in {\layers,...,1} {%
        \foreach \y in {0, ..., \cgs} {%
            \IfFileExists{./res_\x layers_cg\y_N=3.pdf}{%
                %%%%%%%%%%%%%%%%%%%%%%%%%%%%%%%%%%%%%%%%%%%%%%%%%%%%%%%%%%%%%%%%%%%%%%%%%%%%%%%%%%%%%%%%%%%%%%%%
                % Uncomment any of the three below to generate for `residual', `displacement error' or `solution'%
                %%%%%%%%%%%%%%%%%%%%%%%%%%%%%%%%%%%%%%%%%%%%%%%%%%%%%%%%%%%%%%%%%%%%%%%%%%%%%%%%%%%%%%%%%%%%%%%%
                %\includegraphics[width=0.5\textwidth]{res_\x layers_cg\y_\method.pdf}%
                \includegraphics[width=0.5\textwidth]{disp_error_\x layers_cg\y_\method.pdf}%
                %\includegraphics[width=0.5\textwidth]{sol_\x layers_cg\y_\method.pdf}%
                \ifodd\thefigs{\linebreak}\else{}\fi%
                \stepcounter{figs}%
            }{}%
        }
        }  
    %%%%%%%%%%%%%%%%%%%%%%%%%%%%%%%%%%%%%%%%%%%%%%%%%%%%%%%%%%%%%%%%%%%%%%%%%%%%%%%%%%%%%%%%%%%%%%%%
    % The output will be `visualization.pdf', so be sure to rename it to avoid overwritting
    %%%%%%%%%%%%%%%%%%%%%%%%%%%%%%%%%%%%%%%%%%%%%%%%%%%%%%%%%%%%%%%%%%%%%%%%%%%%%%%%%%%%%%%%%%%%%%%%
\end{document}